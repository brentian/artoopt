%%
% Copyright (c) 2017 - 2020, Pascal Wagler;
% Copyright (c) 2014 - 2020, John MacFarlane
%
% All rights reserved.
%
% Redistribution and use in source and binary forms, with or without
% modification, are permitted provided that the following conditions
% are met:
%
% - Redistributions of source code must retain the above copyright
% notice, this list of conditions and the following disclaimer.
%
% - Redistributions in binary form must reproduce the above copyright
% notice, this list of conditions and the following disclaimer in the
% documentation and/or other materials provided with the distribution.
%
% - Neither the name of John MacFarlane nor the names of other
% contributors may be used to endorse or promote products derived
% from this software without specific prior written permission.
%
% THIS SOFTWARE IS PROVIDED BY THE COPYRIGHT HOLDERS AND CONTRIBUTORS
% "AS IS" AND ANY EXPRESS OR IMPLIED WARRANTIES, INCLUDING, BUT NOT
% LIMITED TO, THE IMPLIED WARRANTIES OF MERCHANTABILITY AND FITNESS
% FOR A PARTICULAR PURPOSE ARE DISCLAIMED. IN NO EVENT SHALL THE
% COPYRIGHT OWNER OR CONTRIBUTORS BE LIABLE FOR ANY DIRECT, INDIRECT,
% INCIDENTAL, SPECIAL, EXEMPLARY, OR CONSEQUENTIAL DAMAGES (INCLUDING,
% BUT NOT LIMITED TO, PROCUREMENT OF SUBSTITUTE GOODS OR SERVICES;
% LOSS OF USE, DATA, OR PROFITS; OR BUSINESS INTERRUPTION) HOWEVER
% CAUSED AND ON ANY THEORY OF LIABILITY, WHETHER IN CONTRACT, STRICT
% LIABILITY, OR TORT (INCLUDING NEGLIGENCE OR OTHERWISE) ARISING IN
% ANY WAY OUT OF THE USE OF THIS SOFTWARE, EVEN IF ADVISED OF THE
% POSSIBILITY OF SUCH DAMAGE.
%%

%%
% This is the Eisvogel pandoc LaTeX template.
%
% For usage information and examples visit the official GitHub page:
% https://github.com/Wandmalfarbe/pandoc-latex-template
%%


% @modified: Chuwen <chuwzhang@gmail.com>
% Options for packages loaded elsewhere
\PassOptionsToPackage{unicode}{hyperref}
\PassOptionsToPackage{hyphens}{url}
\PassOptionsToPackage{dvipsnames,svgnames*,x11names*,table}{xcolor}
%
\documentclass[
  10pt,
  a4paper,
,tablecaptionabove
]{scrartcl}
\usepackage{lmodern}
\usepackage{setspace}
\setstretch{1.2}
\usepackage{amssymb,amsmath}
\usepackage{ifxetex,ifluatex}
\ifnum 0\ifxetex 1\fi\ifluatex 1\fi=0 % if pdftex
  \usepackage[T1]{fontenc}
  \usepackage[utf8]{inputenc}
  \usepackage{textcomp} % provide euro and other symbols
\else % if luatex or xetex
  \usepackage{unicode-math}
  \defaultfontfeatures{Scale=MatchLowercase}
  \defaultfontfeatures[\rmfamily]{Ligatures=TeX,Scale=1}
\fi
% Use upquote if available, for straight quotes in verbatim environments
\IfFileExists{upquote.sty}{\usepackage{upquote}}{}
\IfFileExists{microtype.sty}{% use microtype if available
  \usepackage[]{microtype}
  \UseMicrotypeSet[protrusion]{basicmath} % disable protrusion for tt fonts
}{}
\makeatletter
\@ifundefined{KOMAClassName}{% if non-KOMA class
  \IfFileExists{parskip.sty}{%
    \usepackage{parskip}
  }{% else
    \setlength{\parindent}{0pt}
    \setlength{\parskip}{6pt plus 2pt minus 1pt}}
}{% if KOMA class
  \KOMAoptions{parskip=half}}
\makeatother
\usepackage{xcolor}
\definecolor{default-linkcolor}{HTML}{A50000}
\definecolor{default-filecolor}{HTML}{A50000}
\definecolor{default-citecolor}{HTML}{4077C0}
\definecolor{default-urlcolor}{HTML}{4077C0}
\IfFileExists{xurl.sty}{\usepackage{xurl}}{} % add URL line breaks if available
\IfFileExists{bookmark.sty}{\usepackage{bookmark}}{\usepackage{hyperref}}
\hypersetup{
  pdftitle={QAP},
  pdfauthor={Chuwen Zhang},
  hidelinks,
  breaklinks=true,
  pdfcreator={LaTeX via pandoc with the Eisvogel template}}
\urlstyle{same} % disable monospaced font for URLs
\usepackage[margin=2.5cm,includehead=true,includefoot=true,centering,]{geometry}
\usepackage{color}
\usepackage{fancyvrb}
\newcommand{\VerbBar}{|}
\newcommand{\VERB}{\Verb[commandchars=\\\{\}]}
\DefineVerbatimEnvironment{Highlighting}{Verbatim}{commandchars=\\\{\}}
% Add ',fontsize=\small' for more characters per line
\newenvironment{Shaded}{}{}
\newcommand{\AlertTok}[1]{\textcolor[rgb]{1.00,0.00,0.00}{\textbf{#1}}}
\newcommand{\AnnotationTok}[1]{\textcolor[rgb]{0.38,0.63,0.69}{\textbf{\textit{#1}}}}
\newcommand{\AttributeTok}[1]{\textcolor[rgb]{0.49,0.56,0.16}{#1}}
\newcommand{\BaseNTok}[1]{\textcolor[rgb]{0.25,0.63,0.44}{#1}}
\newcommand{\BuiltInTok}[1]{#1}
\newcommand{\CharTok}[1]{\textcolor[rgb]{0.25,0.44,0.63}{#1}}
\newcommand{\CommentTok}[1]{\textcolor[rgb]{0.38,0.63,0.69}{\textit{#1}}}
\newcommand{\CommentVarTok}[1]{\textcolor[rgb]{0.38,0.63,0.69}{\textbf{\textit{#1}}}}
\newcommand{\ConstantTok}[1]{\textcolor[rgb]{0.53,0.00,0.00}{#1}}
\newcommand{\ControlFlowTok}[1]{\textcolor[rgb]{0.00,0.44,0.13}{\textbf{#1}}}
\newcommand{\DataTypeTok}[1]{\textcolor[rgb]{0.56,0.13,0.00}{#1}}
\newcommand{\DecValTok}[1]{\textcolor[rgb]{0.25,0.63,0.44}{#1}}
\newcommand{\DocumentationTok}[1]{\textcolor[rgb]{0.73,0.13,0.13}{\textit{#1}}}
\newcommand{\ErrorTok}[1]{\textcolor[rgb]{1.00,0.00,0.00}{\textbf{#1}}}
\newcommand{\ExtensionTok}[1]{#1}
\newcommand{\FloatTok}[1]{\textcolor[rgb]{0.25,0.63,0.44}{#1}}
\newcommand{\FunctionTok}[1]{\textcolor[rgb]{0.02,0.16,0.49}{#1}}
\newcommand{\ImportTok}[1]{#1}
\newcommand{\InformationTok}[1]{\textcolor[rgb]{0.38,0.63,0.69}{\textbf{\textit{#1}}}}
\newcommand{\KeywordTok}[1]{\textcolor[rgb]{0.00,0.44,0.13}{\textbf{#1}}}
\newcommand{\NormalTok}[1]{#1}
\newcommand{\OperatorTok}[1]{\textcolor[rgb]{0.40,0.40,0.40}{#1}}
\newcommand{\OtherTok}[1]{\textcolor[rgb]{0.00,0.44,0.13}{#1}}
\newcommand{\PreprocessorTok}[1]{\textcolor[rgb]{0.74,0.48,0.00}{#1}}
\newcommand{\RegionMarkerTok}[1]{#1}
\newcommand{\SpecialCharTok}[1]{\textcolor[rgb]{0.25,0.44,0.63}{#1}}
\newcommand{\SpecialStringTok}[1]{\textcolor[rgb]{0.73,0.40,0.53}{#1}}
\newcommand{\StringTok}[1]{\textcolor[rgb]{0.25,0.44,0.63}{#1}}
\newcommand{\VariableTok}[1]{\textcolor[rgb]{0.10,0.09,0.49}{#1}}
\newcommand{\VerbatimStringTok}[1]{\textcolor[rgb]{0.25,0.44,0.63}{#1}}
\newcommand{\WarningTok}[1]{\textcolor[rgb]{0.38,0.63,0.69}{\textbf{\textit{#1}}}}

% Workaround/bugfix from jannick0.
% See https://github.com/jgm/pandoc/issues/4302#issuecomment-360669013)
% or https://github.com/Wandmalfarbe/pandoc-latex-template/issues/2
%
% Redefine the verbatim environment 'Highlighting' to break long lines (with
% the help of fvextra). Redefinition is necessary because it is unlikely that
% pandoc includes fvextra in the default template.
\usepackage{fvextra}
\DefineVerbatimEnvironment{Highlighting}{Verbatim}{breaklines,fontsize=\small,commandchars=\\\{\}}

\usepackage{longtable,booktabs}
% Correct order of tables after \paragraph or \subparagraph
\usepackage{etoolbox}
\makeatletter
\patchcmd\longtable{\par}{\if@noskipsec\mbox{}\fi\par}{}{}
\makeatother
% Allow footnotes in longtable head/foot
\IfFileExists{footnotehyper.sty}{\usepackage{footnotehyper}}{\usepackage{footnote}}
\makesavenoteenv{longtable}
% add backlinks to footnote references, cf. https://tex.stackexchange.com/questions/302266/make-footnote-clickable-both-ways
\usepackage{footnotebackref}
\setlength{\emergencystretch}{3em}  % prevent overfull lines
\providecommand{\tightlist}{%
  \setlength{\itemsep}{0pt}\setlength{\parskip}{0pt}}
\setcounter{secnumdepth}{3}

% Make use of float-package and set default placement for figures to H.
% The option H means 'PUT IT HERE' (as  opposed to the standard h option which means 'You may put it here if you like').
\usepackage{float}
\floatplacement{figure}{H}


\usepackage[UTF8, heading=true]{ctex}
\definecolor{tufeijilk}{RGB}{68,87,151}
\hypersetup{colorlinks=true,linkcolor=tufeijilk,urlcolor=cyan}
\newlength{\cslhangindent}
\setlength{\cslhangindent}{1.5em}
\newenvironment{cslreferences}%
  {}%
  {\par}

\title{QAP}
\author{Chuwen Zhang}
\date{}



%%
%% added
%%

%
% language specification
%
% If no language is specified, use English as the default main document language.
%

\ifnum 0\ifxetex 1\fi\ifluatex 1\fi=0 % if pdftex
  \usepackage[shorthands=off,main=english]{babel}
\else
  % @update, do not use sfdefault,
  % @chuwen, 20200711
  %   % % Workaround for bug in Polyglossia that breaks `\familydefault` when `\setmainlanguage` is used.
  % % See https://github.com/Wandmalfarbe/pandoc-latex-template/issues/8
  % % See https://github.com/reutenauer/polyglossia/issues/186
  % % See https://github.com/reutenauer/polyglossia/issues/127
  % \renewcommand*\familydefault{\sfdefault}
  %   % load polyglossia as late as possible as it *could* call bidi if RTL lang (e.g. Hebrew or Arabic)
  \usepackage{polyglossia}
  \setmainlanguage[]{english}
\fi



%
% for the background color of the title page
%

%
% break urls
%
\PassOptionsToPackage{hyphens}{url}

%
% When using babel or polyglossia with biblatex, loading csquotes is recommended
% to ensure that quoted texts are typeset according to the rules of your main language.
%
\usepackage{csquotes}

%
% captions
%
\definecolor{caption-color}{HTML}{777777}
\usepackage[font={stretch=1.2}, textfont={color=caption-color}, position=top, skip=4mm, labelfont=bf, singlelinecheck=false, justification=raggedright]{caption}
\setcapindent{0em}

%
% blockquote
%
\definecolor{blockquote-border}{RGB}{221,221,221}
\definecolor{blockquote-text}{RGB}{119,119,119}
\usepackage{mdframed}
\newmdenv[rightline=false,bottomline=false,topline=false,linewidth=3pt,linecolor=blockquote-border,skipabove=\parskip]{customblockquote}
\renewenvironment{quote}{\begin{customblockquote}\list{}{\rightmargin=0em\leftmargin=0em}%
\item\relax\color{blockquote-text}\ignorespaces}{\unskip\unskip\endlist\end{customblockquote}}

%
% heading color
%
\definecolor{heading-color}{RGB}{40,40,40}
\addtokomafont{section}{\color{heading-color}}
% When using the classes report, scrreprt, book,
% scrbook or memoir, uncomment the following line.
%\addtokomafont{chapter}{\color{heading-color}}

%
% variables for title and author
%
\usepackage{titling}
\title{QAP}
\author{Chuwen Zhang}

%
% tables
%

\definecolor{table-row-color}{HTML}{F5F5F5}
\definecolor{table-rule-color}{HTML}{999999}

%\arrayrulecolor{black!40}
\arrayrulecolor{table-rule-color}     % color of \toprule, \midrule, \bottomrule
\setlength\heavyrulewidth{0.3ex}      % thickness of \toprule, \bottomrule
\renewcommand{\arraystretch}{1.3}     % spacing (padding)


%
% remove paragraph indention
%
\setlength{\parindent}{0pt}
\setlength{\parskip}{6pt plus 2pt minus 1pt}
\setlength{\emergencystretch}{3em}  % prevent overfull lines

%
%
% Listings
%
%


%
% header and footer
%
\usepackage{fancyhdr}

\fancypagestyle{eisvogel-header-footer}{
  \fancyhead{}
  \fancyfoot{}
  \lhead[]{QAP}
  \chead[]{}
  \rhead[QAP]{}
  \lfoot[\thepage]{Chuwen Zhang}
  \cfoot[]{}
  \rfoot[Chuwen Zhang]{\thepage}
  \renewcommand{\headrulewidth}{0.4pt}
  \renewcommand{\footrulewidth}{0.4pt}
}
\pagestyle{eisvogel-header-footer}

%%
%% end added
%%

\begin{document}

%%
%% begin titlepage
%%

%%
%% end titlepage
%%



\hypertarget{qap-the-problem}{%
\section{QAP, the problem}\label{qap-the-problem}}

QAP, and alternative descriptions, see
\protect\hyperlink{ref-jiang_l_p-norm_2016}{1}

\[\begin{aligned}
&\min_X f(X) = \textrm{tr}(A^\top XB X^\top)  \\
& = \textrm{tr}(X^\top A^\top XB) & x = \textrm{vec}(X)\\
& = \left <\textrm{vec}(X),  \textrm{vec}(A^\top X B )  \right > \\
& = \left <\textrm{vec}(X), B^\top \otimes A^\top \cdot \textrm{vec}(X)  \right > \\ 
& = x^\top (B^\top \otimes A^\top) x\\ 
\mathbf{s.t.} & \\ 
&X \in \Pi_{n}
\end{aligned}\]

is the optimization problem on permutation matrices:

\[ \Pi_{n}=\left\{X \in \mathbb R ^{n \times n} \mid X e =X^{\top} e = e , X_{i j} \in\{0,1\}\right\}\]

The convex hull of permutation matrices, the Birkhoff polytope, is
defined:

\[D _{n}=\left\{X \in \mathbb R ^{n \times n} \mid X e =X^{\top} e = e , X \geq 0 \right\}\]

for the constraints, also equivalently: \[\begin{aligned}
& \textrm{tr}(XX^\top) = \left <x, x \right >_F= n, X \in D_{n}
\end{aligned}\]

\hypertarget{differentiation}{%
\subsection{Differentiation}\label{differentiation}}

\[\begin{aligned}
&  \nabla f = A^\top XB + AXB^\top \\
& \nabla \textrm{tr}(XX^\top) = 2X
\end{aligned}\]

\hypertarget{mathscr-l_p-regularization}{%
\section{\texorpdfstring{\(\mathscr L_p\)
regularization}{\textbackslash mathscr L\_p regularization}}\label{mathscr-l_p-regularization}}

various form of regularized problem:

\begin{itemize}
\item
  \(\mathscr L_0\): \(f(X) + \sigma ||X||_0\) is exact to the original
  problem for efficiently large \(\sigma\)
  \protect\hyperlink{ref-jiang_l_p-norm_2016}{1}, but the problem itself
  is still NP-hard.
\item
  \(\mathscr L_p\): also suggested by
  \protect\hyperlink{ref-jiang_l_p-norm_2016}{1}, good in the sense:

  \begin{itemize}
  \tightlist
  \item
    strongly concave and the global optimizer must be at vertices
  \item
    \textbf{local optimizer is a permutation matrix} if
    \(\sigma, \epsilon\) satisfies some condition. Also, there is a
    lower bound for nonzero entries of the KKT points
  \end{itemize}
\end{itemize}

\[\min _{X \in D _{n}} F_{\sigma, p, \epsilon}(X):=f(X)+\sigma\|X+\epsilon 1 \|_{p}^{p}\]

\begin{itemize}
\tightlist
\item
  \(\mathscr L_2\), and is based on the fact that
  \(\Pi_n = D_n \bigcap \{X:\textrm{tr}(XX^\top) = n\}\),
  \protect\hyperlink{ref-xia_efficient_2010}{2}
\end{itemize}

\[\min_Xf(X)+\mu_{0} \cdot \textrm{tr} \left(X X^{\top}\right)\]

\hypertarget{mathscr-l_2}{%
\subsection{\texorpdfstring{\(\mathscr L_2\)}{\textbackslash mathscr L\_2}}\label{mathscr-l_2}}

\hypertarget{naive}{%
\subsubsection{naive}\label{naive}}

\[\begin{aligned}
&\min_X\textrm{tr}(A^\top XB X^\top) + \mu_0 \cdot \textrm{tr}(X X^{\top}) \\
= & x^\top (B^\top \otimes A^\top + \mu\cdot  \mathbf e_{n\times n}) x\\ 
\end{aligned} \] this implies a LD-like method. (but not exactly)

\hypertarget{a-better-naive}{%
\subsubsection{a better naive}\label{a-better-naive}}

\[\begin{aligned}
&\min_X \mathbf{tr}(M^\top SM) \\&M=\begin{pmatrix}XB\\AX\end{pmatrix},\; S = \begin{pmatrix}\bf{0} & \frac{1}{2}\mathbf{I}\\\frac{1}{2}\mathbf{I} & \bf{0} \end{pmatrix} 
\end{aligned}\] factorizing matrix \(S\) by scale factor \(\delta\)
\[R^\top R = S + \delta I\]

\hypertarget{mathscr-l_2-mathscr-l_1-penalized-formulation}{%
\subsection{\texorpdfstring{\(\mathscr L_2\) + \(\mathscr L_1\)
penalized
formulation}{\textbackslash mathscr L\_2 + \textbackslash mathscr L\_1 penalized formulation}}\label{mathscr-l_2-mathscr-l_1-penalized-formulation}}

Motivated by the formulation using trace:

\[\begin{aligned}
& \min_X  \textrm{tr}(A^\top XB X^\top) \\
\mathbf{s.t.} &\\
&   \textrm{tr}(XX^\top ) -  n = 0 \\
& X \in D_n
\end{aligned}\]

using absolute value of \(\mathscr L_2\) penalty and by the factor that
\(\forall X \in D_n ,\; \textrm{tr}(XX^\top)\le n\), we have:

\[\begin{aligned}
F_{\mu} & =  f  + \mu\cdot | \textrm{tr}(XX^\top ) -  n| \\
 &= \textrm{tr}(A^\top XB X^\top)  + \mu\cdot n - \mu\cdot \textrm{tr}(XX^\top )
\end{aligned}\]

For sufficiently large penalty parameter \(\mu\), the problem solves the
original problem.

The penalty method is very likely to become a concave function (even if
the original one is convex), and thus it cannot be directly solved by
conic solver.

\hypertarget{projected-gradient}{%
\subsubsection{Projected gradient}\label{projected-gradient}}

\begin{Shaded}
\begin{Highlighting}[]
\CommentTok{\# code see}
\NormalTok{qap.models.qap\_model\_l2.l2\_exact\_penalty\_gradient\_proj}
\end{Highlighting}
\end{Shaded}

Suppose we do projection on the penalized problem \(F_\mu\)

\hypertarget{derivatives}{%
\paragraph{derivatives}\label{derivatives}}

\[\begin{aligned}
& \nabla_X F_\mu  = A^\top XB + AXB^\top - 2\mu X \\
& \nabla_\mu F_\mu  = n - \textrm{tr}(XX^\top) \\
& \nabla_\Lambda F_\mu  = - X
\end{aligned}\]

\hypertarget{projected-derivative}{%
\paragraph{projected derivative}\label{projected-derivative}}

problem \emph{PD}, a quadratic program

\[\begin{aligned}
&\min_D ||\nabla F_\mu + D ||_F^2  \\
\mathbf{s.t.} & \\
&D e = D^\top e = 0 \\ 
&D_{ij} \ge 0 \quad \textsf{if: } X_{ij} = 0\\
\end{aligned}\]

or equivalently, a linear program

\[\begin{aligned}
&\min_D \nabla F_\mu \bullet D   \\
\mathbf{s.t.} & \\
&D e = D^\top e = 0 \\ 
&D_{ij} \ge 0 \quad \textsf{if: } X_{ij} = 0\\ 
&||D||\le 1 \\
\end{aligned}\]

\begin{itemize}
\tightlist
\item
  There is no degeneracy, great.
\end{itemize}

\hypertarget{remark}{%
\subsubsection{Remark}\label{remark}}

\hypertarget{integrality-of-the-solution}{%
\paragraph{integrality of the
solution}\label{integrality-of-the-solution}}

Computational results show that the \emph{residue of the trace}:
\(|n - \textrm{tr}(XX^\top)|\) is almost zero, this means the algorithm
converges to an integral solution. (even without any tuning of penalty
parameter \(\mu\))

Prove that it is exact if \(\mu\) is sufficiently large, the model
converges to an integral solution.

\textbf{PF. outline} \textbf{a}. the model uses exact penalty function,
as \(\mu\), actually \(\{\mu_k\}\) become sufficiently large, the
penalty method solves the original problem. \textbf{b}. if \(\{\mu_k\}\)
become sufficiently large, the penalized objective will be concave, so
that the optimal solution should be attained at the vertices.

\hypertarget{quality-of-the-solution}{%
\paragraph{quality of the solution}\label{quality-of-the-solution}}

it is however hard to find a global optimum, and gradient projection as
defined above converges to a local integral solution and then stops, see
instances with gap \textgreater{} 10\%.

\hypertarget{analytic-representation-for-projection}{%
\paragraph{analytic representation for
projection}\label{analytic-representation-for-projection}}

in projected gradient method, let the space of \(D\), (\(e\) is the
vector of 1s)

\[\mathcal D = \{D\in\mathbb{R}^{n\times n} : \; D e = D^\top e = 0;\; D_{ij} = 0,\;\forall  (i,j) \in M \}\]

is there a way to formulate the set for \(F\) such that
\(\left <F, D \right>_F = 0, \; \forall D\in \mathcal D\), can we find
an analytic representation?

\hypertarget{what-if-projection-is-zero}{%
\paragraph{*what if projection is
zero?}\label{what-if-projection-is-zero}}

dual problem for \(PD\)

\begin{itemize}
\tightlist
\item
  \(\alpha,\beta,\Lambda\) are Lagrange multipliers, \(\mathbf I\) is
  the identity matrix for active constraints of the \(X \ge 0\) where
  \(\mathbf I_{ij} = 1\) if \(X_{ij} = 0\)
\end{itemize}

\[\begin{aligned}
& L_d = 1/2\cdot ||\nabla F_\mu + D ||_F^2 - \alpha^\top De - \beta^\top D^\top e -\Lambda \circ \mathbf I \bullet D\\
\mathsf{KKT:} & \\
& \nabla F+D - ae^\top - e\beta^\top -\Lambda \circ \mathbf{I} = 0 \\
& \Lambda \ge 0
\end{aligned}\]

Suppose at iteration \(k\) projected gradient \(D_k = 0\), then the KKT
condition for

We relax one condition for active inequality for some
\(e = (i,j), e \in M\) such that \(X_e =0\), a new optimal direction for
problem PD is achieved at \(\hat D\), we have:

\[\begin{aligned}
 & \hat D_{ij} - (\alpha_i + \beta_j) + (\hat \alpha_i + \hat \beta_j) - \Lambda_{ij} = 0, \quad e = (i,j) \\
\end{aligned}\]

\hypertarget{computational-results}{%
\section{Computational Results}\label{computational-results}}

The experiments are done on dataset of
\href{http://anjos.mgi.polymtl.ca/qaplib/}{QAPLIB}, also see paper
{[}\protect\hyperlink{ref-burkard1997qaplib}{3}{]}

The \(\mathscr L_2\) penalized
\protect\hyperlink{mathscr-l_2--mathscr-l_1-penalized-formulation}{formulation}
with code
\texttt{qap.models.qap\_model\_l2.l2\_exact\_penalty\_gradient\_proj},
solved by gradient projection of module
\texttt{qap.models.qap\_gradient\_proj} can solve almost all instances,
except for very large ones (\(\ge\) 256), it should be better since it
now uses Mosek as backend to solve orthogonal projections, line search
for step-size, and so on.

current benchmark

\begin{longtable}[]{@{}lllll@{}}
\caption{L\_2 + L\_1 penalized gradient projection}\tabularnewline
\toprule
\begin{minipage}[b]{0.11\columnwidth}\raggedright
\strut
\end{minipage} & \begin{minipage}[b]{0.20\columnwidth}\raggedright
value\strut
\end{minipage} & \begin{minipage}[b]{0.13\columnwidth}\raggedright
rel\_gap\strut
\end{minipage} & \begin{minipage}[b]{0.15\columnwidth}\raggedright
trace\_res\strut
\end{minipage} & \begin{minipage}[b]{0.15\columnwidth}\raggedright
runtime\strut
\end{minipage}\tabularnewline
\midrule
\endfirsthead
\toprule
\begin{minipage}[b]{0.11\columnwidth}\raggedright
\strut
\end{minipage} & \begin{minipage}[b]{0.20\columnwidth}\raggedright
value\strut
\end{minipage} & \begin{minipage}[b]{0.13\columnwidth}\raggedright
rel\_gap\strut
\end{minipage} & \begin{minipage}[b]{0.15\columnwidth}\raggedright
trace\_res\strut
\end{minipage} & \begin{minipage}[b]{0.15\columnwidth}\raggedright
runtime\strut
\end{minipage}\tabularnewline
\midrule
\endhead
\begin{minipage}[t]{0.11\columnwidth}\raggedright
count\strut
\end{minipage} & \begin{minipage}[t]{0.20\columnwidth}\raggedright
115.000\strut
\end{minipage} & \begin{minipage}[t]{0.13\columnwidth}\raggedright
114.000\strut
\end{minipage} & \begin{minipage}[t]{0.15\columnwidth}\raggedright
115.000\strut
\end{minipage} & \begin{minipage}[t]{0.15\columnwidth}\raggedright
115.000\strut
\end{minipage}\tabularnewline
\begin{minipage}[t]{0.11\columnwidth}\raggedright
mean\strut
\end{minipage} & \begin{minipage}[t]{0.20\columnwidth}\raggedright
58087685.641\strut
\end{minipage} & \begin{minipage}[t]{0.13\columnwidth}\raggedright
0.300\strut
\end{minipage} & \begin{minipage}[t]{0.15\columnwidth}\raggedright
2.068\strut
\end{minipage} & \begin{minipage}[t]{0.15\columnwidth}\raggedright
12.905\strut
\end{minipage}\tabularnewline
\begin{minipage}[t]{0.11\columnwidth}\raggedright
std\strut
\end{minipage} & \begin{minipage}[t]{0.20\columnwidth}\raggedright
201707868.192\strut
\end{minipage} & \begin{minipage}[t]{0.13\columnwidth}\raggedright
0.544\strut
\end{minipage} & \begin{minipage}[t]{0.15\columnwidth}\raggedright
20.853\strut
\end{minipage} & \begin{minipage}[t]{0.15\columnwidth}\raggedright
84.748\strut
\end{minipage}\tabularnewline
\begin{minipage}[t]{0.11\columnwidth}\raggedright
min\strut
\end{minipage} & \begin{minipage}[t]{0.20\columnwidth}\raggedright
0.000\strut
\end{minipage} & \begin{minipage}[t]{0.13\columnwidth}\raggedright
0.000\strut
\end{minipage} & \begin{minipage}[t]{0.15\columnwidth}\raggedright
0.000\strut
\end{minipage} & \begin{minipage}[t]{0.15\columnwidth}\raggedright
0.039\strut
\end{minipage}\tabularnewline
\begin{minipage}[t]{0.11\columnwidth}\raggedright
25\%\strut
\end{minipage} & \begin{minipage}[t]{0.20\columnwidth}\raggedright
3962.001\strut
\end{minipage} & \begin{minipage}[t]{0.13\columnwidth}\raggedright
0.024\strut
\end{minipage} & \begin{minipage}[t]{0.15\columnwidth}\raggedright
0.000\strut
\end{minipage} & \begin{minipage}[t]{0.15\columnwidth}\raggedright
0.125\strut
\end{minipage}\tabularnewline
\begin{minipage}[t]{0.11\columnwidth}\raggedright
50\%\strut
\end{minipage} & \begin{minipage}[t]{0.20\columnwidth}\raggedright
97330.031\strut
\end{minipage} & \begin{minipage}[t]{0.13\columnwidth}\raggedright
0.091\strut
\end{minipage} & \begin{minipage}[t]{0.15\columnwidth}\raggedright
0.000\strut
\end{minipage} & \begin{minipage}[t]{0.15\columnwidth}\raggedright
0.215\strut
\end{minipage}\tabularnewline
\begin{minipage}[t]{0.11\columnwidth}\raggedright
75\%\strut
\end{minipage} & \begin{minipage}[t]{0.20\columnwidth}\raggedright
3207963.510\strut
\end{minipage} & \begin{minipage}[t]{0.13\columnwidth}\raggedright
0.255\strut
\end{minipage} & \begin{minipage}[t]{0.15\columnwidth}\raggedright
0.000\strut
\end{minipage} & \begin{minipage}[t]{0.15\columnwidth}\raggedright
1.584\strut
\end{minipage}\tabularnewline
\begin{minipage}[t]{0.11\columnwidth}\raggedright
max\strut
\end{minipage} & \begin{minipage}[t]{0.20\columnwidth}\raggedright
1289655958.000\strut
\end{minipage} & \begin{minipage}[t]{0.13\columnwidth}\raggedright
4.000\strut
\end{minipage} & \begin{minipage}[t]{0.15\columnwidth}\raggedright
223.278\strut
\end{minipage} & \begin{minipage}[t]{0.15\columnwidth}\raggedright
885.492\strut
\end{minipage}\tabularnewline
\bottomrule
\end{longtable}

\hypertarget{reference}{%
\section*{Reference}\label{reference}}
\addcontentsline{toc}{section}{Reference}

\hypertarget{refs}{}
\begin{cslreferences}
\leavevmode\hypertarget{ref-jiang_l_p-norm_2016}{}%
{[}1{]} B. Jiang, Y.-F. Liu, and Z. Wen, ``L\_p-norm regularization
algorithms for optimization over permutation matrices,'' \emph{SIAM
Journal on Optimization}, vol. 26, no. 4, pp. 2284--2313, 2016.

\leavevmode\hypertarget{ref-xia_efficient_2010}{}%
{[}2{]} Y. Xia, ``An efficient continuation method for quadratic
assignment problems,'' \emph{Computers \& Operations Research}, vol. 37,
no. 6, pp. 1027--1032, 2010.

\leavevmode\hypertarget{ref-burkard1997qaplib}{}%
{[}3{]} R. E. Burkard, S. E. Karisch, and F. Rendl, ``QAPLIB--a
quadratic assignment problem library,'' \emph{Journal of Global
optimization}, vol. 10, no. 4, pp. 391--403, 1997.
\end{cslreferences}

\end{document}
